\chapter{Introduction}

\section{Motivation}
\label{motivation}
This project has been based upon the idea of giving the user the power to visually and interactively explore all the paths generated by a path tracer during rendering. In the very first vision, a user should have been able to select a portion of a surface of the 3d scene the tracer has been run upon and see the paths that bounce there with a bunch of useful data. To be able to do that the whole set of paths shoot by a tracer are needed: by the very stochastic nature of a path tracer, it is impossible to determine which paths will end up bouncing where without resolving them all first. That is why it has been decided it was essential to store data about each path during the rendering process. To make the tool usable in most possible use cases, it had to be able to plug into an existing path tracer and this lead to the conception of the tool as a two software pieces suite: a \textit{data gatherer library} called \texttt{gatherer} and a \textit{visualization client} called \texttt{gathererclient}.

%Now this “useful data” was not extremely well-defined during those early stages, so most of the efforts have been directed to the very essential: rendering the requested paths keeping interactivity.

\section{Technology}
\label{technology}

C++ does not have a built-in \texttt{half} type so an IEEE 754 compatible header only library\footnote{\url{http://half.sourceforge.net/}} has been used thoroughly the written software.